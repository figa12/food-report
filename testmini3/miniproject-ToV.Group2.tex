\documentclass[a4paper,12pt]{memoir}
\usepackage{framed}
\usepackage{enumitem}
\usepackage{todonotes}

%Commands
\newcommand{\testdesign}[4]
{
\begin{framed}
\begin{description}
\item \textbf{Identifier:} #1
\item \textbf{Features to be tested:} #2
\item \textbf{Approach:} #3
%\item \textbf{Test case identification:} #4
\item \textbf{Pass/fail criteria:} #4
\end{description}
\end{framed}
}

\title{Miniproject in Test and Verification - Part 1}
\author{Jacob Karstensensen Wortmann\\Sam Sepstrup Olesen\\Nicklas Andersen\\sw805f14}

\begin{document}
\maketitle %project description
We will in our project make an application that functions as a digital cookbook. The user is able to search for recipes by different search functions: by ingredients and by the name of the recipe. Users can share the recipes with friends and relatives, it is also possible to favourite a recipe. Our application should also support conversion between the imperial and metric system to reach a larger audience. 
We want to create a thin client which does not perform a huge amount of calculation, this means that we can focus on testing the server and the communication layer in the client.

\section{Mutation testing}
\section{Code coverage}

\end{document}