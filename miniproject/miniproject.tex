\documentclass[a4paper,12pt]{memoir}


\title{Software Innovation: Miniproject}
\author{Jacob Karsten Wortmann\\Jesper Riemer Andersen\\Nicklas Andersen\\Sam Sepstrup Olesen\\Simon Reedtz Olesen}

\begin{document}
\maketitle

\section*{Brief description}
The idea of the project is to make an interactive cookbook on the android platform that can give you suggestions of recipes based on chosen ingredients. An advantage of the interactive recipe search based on ingredients could be that whenever you find an interesting ingredient in the store, you can choose this ingredient in the application and quickly get relevant suggestions of recipes and thereby identify which additional ingredients you need to buy. Another use for this functionality is to find interesting recipes that you can make with the ingredients you already have at home or few additional ingredients.

\section*{Challenges}
The main challenge of the project is to make an efficient search algorithm that rivals the other similar applications on the market. The problem with the other existing solutions is that if you for example input ``eggs'' and ``chicken'', you will get many different recipes of how to prepare eggs, and not on how to combine the two ingredients. The overall design of the application can also prove to be a major challenge because we have to fit a large amount of information on a relatively small mobile screen.

\chapter*{Section 1}
% a brief description of your initial project idea at the start of the semester (no more than half a page).

\section*{Problem Domain}

When a user stands in their kitchen or in the supermarket and wants to look up a recipe for dinner, they must be able to access this easily.

Our application focuses on making it easy to access recipes on mobile devices. We want to split up the recipes in core and optional ingredients, in order to optimise the search results the user gets. For this we need to analyse each recipe that will be used in the application.

\section*{Use Context}

The application can be used in different scenarios, for example the user can be at home and type in the items they have in their kitchen to get suggestions for that recipes they can use. The user can also stand in the supermarket and see an item on sale and use the interactive system to find which recipes they can make with that item and what other items they also need to buy. 

\paragraph{Metaphor}

A metaphor for our application could be a cookbook on your mobile phone. 

\paragraph{Focus on the Essentials}

\section*{Affordance}

\begin{description}

\item [Navigation drawer] is a feature in Android, allowing the user to swipe from the left of the screen and open a "drawer" with different views that they can navigate to. Navigation drawer is standard design pattern in Android so this design will behave in a consistent and predictable fashion known to the user.
\item [Action bar] is also a feature in Android. Like the navigation drawer this is also a standard design pattern. This is where the user expects the actions to be, so this is the natural placement of our search function. The action bar also shows that the navigation drawer is available and it can also open it since it is an action. 
\item[Scrolling] is a basic feature in almost every application in order to show more information than what fits on the screen. The scrolling feature is usually indicated by partially displaying the content at the bottom or at either side.

\end{description}

\section*{Evaluation Criteria}

The application must be fast to use, it must also have a low learning curve.

\chapter*{Section 2}
% work-products from using Essence. You can use results from the exercises, but you may also use later material from your project work. The work-products must relate to all four Views in Essence: Paradigm (scenarios, states, prototype), Product (design ideas, technological options), Project (Toulmin structure, other representations of visions, research visions, feature list), and Process (SWOT evaluation and one or both of these: value analysis, use criteria). Work-products in italics are recommended but not mandatory. Provide enough information in this section to illuminate and substantiate your observations in the rest of the presentation.



\chapter*{Section 3}
% description of how your results developed from your initial project idea (the problem you started out to solve) to status at the time of writing this mini-project. Were there any important leaps in your visions, product ideas, or perceptions of use context?


\chapter*{Section 4}
% theoretical evaluation

\end{document}