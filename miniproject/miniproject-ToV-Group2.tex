\documentclass[a4paper,12pt]{memoir}
\usepackage{framed}
\usepackage{enumitem}
\usepackage{todonotes}

\newcommand{\testdesign}[5]
{
\begin{framed}
\begin{description}
\item \textbf{Identifier:} #1
\item \textbf{Features to be tested:} #2
\item \textbf{Approach:} #3
\item \textbf{Test case identification:} #4
\item \textbf{Pass/fail criteria:} #5
\end{description}
\end{framed}
}

\title{Miniproject in Test and Verification - Part 1}
\author{Jacob Karsten Wortmann\\Sam Sepstrup Olesen\\Nicklas Andersen}

\begin{document}
\maketitle
\chapter*{Miniproject}
\section*{1a - Test Design}
\subsection*{Specification}
%The purpose of the test design spec is to organize and describe the testing that needs to be performed on a specific feature. It doesn’t, however, give the detailed cases or the steps to execute to perform the testing.
\testdesign
{ServerTest}
{Add items to shopping list. Items can be added from the shopping list or from a recipe.}
{A shopping list written in hand will be created and then added to the shopping list in the application. Items from the recipe will be added to the shopping list.}
{\begin{itemize}[nolistsep]
\item 
\end{itemize}}
{A pass is when all test cases are run without finding a bug}

\section*{1b - Test cases}

\end{document}