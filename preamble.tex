\documentclass[a4paper,twoside]{memoir}
% Set up encoding
\usepackage[utf8]{inputenc}

% Mulighed for at definere acronyms
\usepackage{acronym}

%package for linenumbers
\usepackage{lineno} 

% Load up bibliography.
\usepackage[authoryear]{natbib}
\setcitestyle{numbers,square}
% Bibliography style.
\bibliographystyle{plainnat}

% Algorithm support.
\usepackage{algorithmic}
\usepackage{algorithm}
\usepackage{subfig}
\usepackage{amsmath}
\usepackage{amsfonts}
% Make algorithms appear as procedures instead.
\floatname{algorithm}{Procedure}
\renewcommand{\algorithmicrequire}{\textbf{Input:}}
\renewcommand{\algorithmicensure}{\textbf{Output:}}

% Image frames.
\setlength{\fboxsep}{0pt}
\setlength{\fboxrule}{0.5pt}

% Also, images.
\usepackage{graphicx}

% tabeller der strækker sig over flere sider
\usepackage{longtable}

% flere tabel-muligheder
\usepackage{multirow}

% bedre enumerate
\usepackage{enumitem}

% Mulighed for if-then-else sætning!
\usepackage{ifthen}

% Todo notes here and there.
% write instead for disable: \usepackage[disable]{todonotes}
\usepackage{todonotes}

% Forbedrede floats.
\usepackage{float}
\usepackage{rotating}

\newsubfloat{figure}

% Special symbols availability.
\usepackage{amssymb}

%Degree symbol
\usepackage{gensymb}

% Wrap figure
\usepackage{wrapfig}

% Remove subsection numbering
\renewcommand{\thesubsection}{}
\makeatletter
\def\@seccntformat#1{\csname #1ignore\expandafter\endcsname\csname the#1\endcsname\quad}
\let\subsectionignore\@gobbletwo
\let\latex@numberline\numberline
\def\numberline#1{\if\relax#1\relax\else\latex@numberline{#1}\fi}
\makeatother

%landscape mode
\usepackage{lscape}

\usepackage{booktabs}
\usepackage{hvfloat}
\usepackage{units}

%Fun with captions
\usepackage{caption}


% Neat-o referencer...o.
\usepackage{bookmark,hyperref}
\usepackage{nameref}

\newcommand{\secref}[1]{Section \ref{#1}}
\newcommand{\chapref}[1]{Chapter \ref{#1}}
\newcommand{\appref}[1]{Appendix \ref{#1}}

% Skriver Appendix foran Appendices
\renewcommand*{\cftpartname}{PART~}
%\renewcommand*{\cftchaptername}{\chaptername~}
\renewcommand*{\cftappendixname}{\appendixname~}
%\renewcommand*{\cftchapteraftersnum}{.}% dot after the number
%\setlength{\cftchapternumwidth}{2em}

% Operationel semantik
\newcommand{\lag}{\langle}
\newcommand{\rag}{\rangle}
\newcommand{\setof}[2]{\ensuremath{\{ #1 \mid #2 \}}}
\newcommand{\set}[1]{\ensuremath{\{ #1 \}}}
\newcommand{\besk}[1]{\ensuremath{\lag #1 \rag}}
\newcommand{\ra}{\rightarrow}
\newcommand{\lra}{\longrightarrow}
\newcommand{\Ra}{\Rightarrow}

% CODE %
\usepackage{listings}
\usepackage{color}
%\usepackage{bera}
\definecolor{gray}{rgb}{0.4,0.4,0.4}
\definecolor{darkblue}{rgb}{0.0,0.0,0.6}
\definecolor{cyan}{rgb}{0.0,0.6,0.6}
\definecolor{dkgreen}{rgb}{0,.6,0}
\definecolor{dkblue}{rgb}{0,0,.6}
\definecolor{dkyellow}{cmyk}{0,0,.8,.3}

\lstset{
  basicstyle=\ttfamily,
  columns=fullflexible,
  showstringspaces=false,
  commentstyle=\color{gray}\upshape,
  basicstyle=\small,
  numberstyle=\footnotesize,
  numbers=left,
  captionpos=b,
  stepnumber=1,
  numbersep=10pt,
  tabsize=2,
  breaklines=true,
}
% Define markup of XML
\lstdefinelanguage{XML}
{
  morestring=[b]",
  morestring=[s]{>}{<},
  morecomment=[s]{<?}{?>},
  identifierstyle=\color{darkblue},
  keywordstyle=\color{cyan},
  morekeywords={id, target, type, category, value, point, correct, rows, width, time}% list your attributes here
}
% Define markup of C#
\lstdefinelanguage{CSharp}[Visual]{C++}
{
	identifierstyle=\color{darkblue},
	commentstyle=\color{green!70!black}\itshape ,
	stringstyle=\color{gray},
	sensitive=true,
	morestring=[b]",
	morestring=[b]',
	morecomment=[l]//,
	morecomment=[n]{/*}{*/}
}

% Define markup of Javascript
\lstdefinelanguage{JavaScript}{
  keywords={typeof, new, true, false, catch, function, return, null, catch, switch, var, if, in, while, do, else, case, break},
  keywordstyle=\color{blue}\bfseries,
  ndkeywords={class, export, boolean, throw, implements, import, this},
  ndkeywordstyle=\color{darkgray}\bfseries,
  identifierstyle=\color{black},
  sensitive=false,
  comment=[l]{//},
  morecomment=[s]{/*}{*/},
  commentstyle=\color{purple}\ttfamily,
  stringstyle=\color{red}\ttfamily,
  morestring=[b]',
  morestring=[b]"
}

% Define markup of Java
\definecolor{dkgreen}{rgb}{0,0.6,0}
\definecolor{gray}{rgb}{0.5,0.5,0.5}
\definecolor{mauve}{rgb}{0.58,0,0.82}
\definecolor{keywordpurple}{RGB}{145, 0, 109}
\definecolor{background}{RGB}{240, 240, 240}
 
\lstset{
  language=java,
  %basicstyle=\footnotesize,       % the size of the fonts that are used for the code
  numbers=left,                   % where to put the line-numbers
  numberstyle=\tiny\color{black},  % the style that is used for the line-numbers
  stepnumber=1,                   % the step between two line-numbers. If it's 1, each line will be numbered 
  numbersep=5pt,                  % how far the line-numbers are from the code
  backgroundcolor=\color{background},  % choose the background color. You must add \usepackage{color}
  showspaces=false,               % show spaces adding particular underscores
  showstringspaces=false,         % underline spaces within strings
  showtabs=false,                 % show tabs within strings adding particular underscores
  frame=single,                   % adds a frame around the code
  rulecolor=\color{black},        % if not set, the frame-color may be changed on line-breaks within not-black text (e.g. comments (green here))
  tabsize=4,                      % sets default tabsize to 4 spaces
  captionpos=b,                   % sets the caption-position to bottom
  breaklines=true,                % sets automatic line breaking
  breakatwhitespace=false,        % sets if automatic breaks should only happen at whitespace
  title=\lstname,                 % show the filename of files included with \lstinputlisting;
                                  % also try caption instead of title
  keywordstyle=\color{keywordpurple}\bfseries,      % keyword style
  commentstyle=\color{dkgreen},   % comment style
  stringstyle=\color{blue},      % string literal style
  escapeinside={\%*}{*)},         % if you want to add a comment within your code
  morekeywords={*,...},           % if you want to add more keywords to the set
  morecomment=[l]//               % set // to register as a comment (for a line)
}

% Define markup of JSON
\colorlet{punct}{red!60!black}
\colorlet{delim}{red!60!black}
\colorlet{numb}{magenta!60!black}
\lstdefinelanguage{json}{
    basicstyle=\footnotesize,
    numbers=left,
    numberstyle=\tiny\color{black},
    identifierstyle=\color{dkgreen},
    stepnumber=1,
    numbersep=5pt,
    showspaces=false,
    showstringspaces=false,
    showtabs=false,
    breaklines=true,
    breakatwhitespace=false,
    tabsize=4,
    rulecolor=\color{black},
    captionpos=b,
    title=\lstname,
    frame=single,
    backgroundcolor=\color{background},
    literate=
     *{0}{{{\color{numb}0}}}{1}
      {1}{{{\color{numb}1}}}{1}
      {2}{{{\color{numb}2}}}{1}
      {3}{{{\color{numb}3}}}{1}
      {4}{{{\color{numb}4}}}{1}
      {5}{{{\color{numb}5}}}{1}
      {6}{{{\color{numb}6}}}{1}
      {7}{{{\color{numb}7}}}{1}
      {8}{{{\color{numb}8}}}{1}
      {9}{{{\color{numb}9}}}{1}
      {:}{{{\color{punct}{:}}}}{1}
      {,}{{{\color{punct}{,}}}}{1}
      {\{}{{{\color{delim}{\{}}}}{1}
      {\}}{{{\color{delim}{\}}}}}{1}
      {[}{{{\color{delim}{[}}}}{1}
      {]}{{{\color{delim}{]}}}}{1},
}

\lstdefinelanguage{phpstyle}{
  language        = php,
  keywordstyle    = \color{dkblue},
  morekeywords    = {function},
  stringstyle     = \color{red}
  }

\lstdefinelanguage{KAPAOOW}{
 sensitive=false,
 keywords={character, characters, action, end, if, then, else, from, to, downto, next, while, loop, use, turn, begins, ends, select, wins, draw, random, of, case, cases, enemy, player, start, skip, attack, types, damage, defend, by, using, message, and, or, is, value, mod},
 identifierstyle=\itshape,
 keywordstyle=\bfseries,
 stringstyle=\normalfont,
 morestring=[b]",
 comment=[l]{//},
 commentstyle=\color{gray}
}

% hack fra nettet.
% http://tex.stackexchange.com/questions/1230/reference-name-of-description-list-item-in-latex
\makeatletter
\let\orgdescriptionlabel\descriptionlabel
\renewcommand*{\descriptionlabel}[1]{
  \let\orglabel\label
  \let\label\@gobble
  \phantomsection
  \edef\@currentlabel{#1}
  %\edef\@currentlabelname{#1}
%  \let\label\orglabel
  \orgdescriptionlabel{#1}
}
\makeatother
% Rettehak. Meget lettere end \checkmark
\newcommand{\yes}{\checkmark}


% Create a new command, HRule, to insert some nice horisontal rules on the title page.
\newcommand{\HRule}{\rule{\linewidth}{0.3mm}}

% New command for two figures, side by side.
\newcommand{\twofigs}[6]
{
	\begin{figure}[H]
		\begin{minipage}[b]{0.5\columnwidth}
		\centering
		\includegraphics[width=0.8\columnwidth]{img/#1}
		\caption{#2\label{#3}}
		\end{minipage}
		\hspace{0.5cm}
		\begin{minipage}[b]{0.5\columnwidth}
		\centering
		\includegraphics[width=0.8\columnwidth]{img/#4}
		\caption{#5\label{#6}}
		\end{minipage}
	\end{figure}
}

% Sørg for at paragrafplads ikke spildes.
\raggedbottom

% Package til at regne forskellen ud mellem 2 labels
\usepackage{refcount}
\newcommand{\pagedifference}[2]{\number\numexpr\getpagerefnumber{#2}+1-\getpagerefnumber{#1}\relax}


% Fancy chapter style

\usepackage{color,calc,graphicx,soul,fourier}
\definecolor{aaublue}{RGB}{33,26,82}
\makeatletter
\newlength\dlf@normtxtw
\setlength\dlf@normtxtw{\textwidth}
\def\myhelvetfont{\def\sfdefault{mdput}}
\newsavebox{\feline@chapter}
\newcommand\feline@chapter@marker[1][4cm]{%
  \sbox\feline@chapter{%
    \resizebox{!}{#1}{\fboxsep=1pt%
      \colorbox{aaublue}{\color{white}\bfseries\sffamily\thechapter}%
    }}%
  \rotatebox{90}{%
    \resizebox{%
      \heightof{\usebox{\feline@chapter}}+\depthof{\usebox{\feline@chapter}}}%
    {!}{\scshape\so\@chapapp}}\quad%
  \raisebox{\depthof{\usebox{\feline@chapter}}}{\usebox{\feline@chapter}}%
}
\newcommand\feline@chm[1][4cm]{%
  \sbox\feline@chapter{\feline@chapter@marker[#1]}%
  \makebox[0pt][l]{% aka \rlap
    \makebox[1cm][r]{\usebox\feline@chapter}%
  }}
\makechapterstyle{daleif1}{
  \renewcommand\chapnamefont{\normalfont\Large\scshape\raggedleft\so}
  \renewcommand\chaptitlefont{\normalfont\huge\bfseries\scshape\color{aaublue}}
  \renewcommand\chapternamenum{}
  \renewcommand\printchaptername{}
  \renewcommand\printchapternum{\null\hfill\feline@chm[2.5cm]\par}
  \renewcommand\afterchapternum{\par\vskip\midchapskip}
  \renewcommand\printchaptertitle[1]{\chaptitlefont\raggedleft ##1\par}
}
\makeatother
\chapterstyle{daleif1}
\setlength\afterchapskip {\onelineskip }
\setlength\beforechapskip {\onelineskip }
\usepackage{lipsum}

%Laver fancy ting her. Noget med at overwrite noget includegraphics for at kunne bruge commands som parameter til den
\makeatletter
\protected\def\includeGraphics{\@testopt\roy@includegraphics{}}
\def\roy@includegraphics[#1]#2{%
  \begingroup
  % Every expandable token in #1 may be expanded here:
  \edef\x{\endgroup\noexpand\includegraphics[#1]}\x{#2}%
}
\makeatother

\newcommand{\theAngle}{90} %vinkel brugt til at rotere med, bliver renewed i \landscapefigure

%Indsætter figure i landscape og roterer billedet efter om det er højre eller venstre side
\newcommand{\landscapefigure}[4]
{
\ifthenelse{\isodd{\thepage}}
{% ulige sidetal = højre side
\renewcommand{\theAngle}{90}
}
{% lige sidetal = venstre side
\renewcommand{\theAngle}{270}
}

\begin{figure}[H]
\centering
\includegraphics[angle=\theAngle ,#1]{img/#2}
\caption{#3}
\label{#4}
\end{figure}
}

\hyphenation{guard-i-an}
