Due to time constraints in our project we have notmanaged to implement all functionalities.

\begin{description}
\item[Shopping list] We wanted to have a shopping list where ingredients could easily be added from recipes and also from the shopping list itself.

\item[Startpage] The application needs some kind of a startpage, right now the application opens in the ingredient search page. The startpage can be a page displaying popular or featured recipes to inspire the user, but an automatic display of popular recipes might become a static page with the same recipes displayed, and featured recipes needs to be maintained. A third option for a startpage can be a welcome message and a small guide or tutorial on how the application works and what it does.

\item[Recipe filters] The user should be able to specify some filters for search results. Examples are filters for vegans and allergies.\todo{Flere eksempler?}

\item[Sharing] You cannot currently share anything in the application. We wanted the ability to share recipes and shopping lists. Sharing can happen through services like SMS, Facebook, or Google+.

\item[Recipe cache] Right now the recipes are downloaded each time they are opened. All the images are already automatically cached in the application, caching recipes also makes sense, since they are probably likely to opened multiple times, especially when they are favourited for later use.

\item[Recipe license] We need to display a license on all the recipes in the application, the application can currently display a license, but the server does not send it.

\item[Conversion] The database and the model currently contains a conversion constant for each unit, but the application does not contain the functionality to convert between units. Right now in the database decilitre is converted to ounces, but usually the imperial system deals with cups.\todo{skal der stå en løsning på det konverterings problem?}

\item[Recipe references] We wanted recipes to be able to have references to other recipes, because ingredient groups might be big enough to be its own recipe. This is currently not possible.

\item[Settings] The application does not have a settings page. Settings could include: Metric or imperial standard option, filtering of recipes e.g. by allergies or vegetarian, revoke access to the user's Google+ account.\todo{flere settings?}

\item[Scaling of recipes] Button to scale the ingredients of a recipe to more people.

\item [Intelligent sort of ingredients in word cloud] Instead of having a static word cloud like we have now, we want the word cloud to update each time a user adds an ingredient to their search. We want the word cloud to update with suggestions based on the ingredients already added. \todo{Skriv mere}

%\item Shoppinglist
%\item Settings
%\item Salt er der som ingrediens? Settings?
%\item Startpage
%\item Conversion
%\item Filter (vegan)
%\item Refs to other recipes
\item [Localization] At the moment the application only supports English. We want to support more languages in the future in order to reach a wider audience.\todo{Skriv mere}
%\item Share recipes and shopping lists (sms, facebook, google+)
\item [Log interesting ingredient searches] It could be very interesting to log what ingredients the users searches for the most. This could identify what ingredients the user search for and maybe help us improve our selection of recipes. It might also be a help for the word cloud to help identify what ingredient combinations use.\todo{Meh?}
%\item Recipe cache on mobile device (Perhaps FIFO)
%\item llicense på recipes
\item [Revoke Google account access]

\end{description}
\todo{skriv future work om disse elementer}