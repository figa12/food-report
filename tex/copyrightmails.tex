\chapter{Copyright clarification}\label{copyrightmails}

\section*{Our question to the lawyers}
\begin{lstlisting}[numbers=none, breakindent=0pt, basicstyle=\ttfamily,keywordstyle=\ttfamily, numberstyle=\ttfamily, commentstyle=\ttfamily, stringstyle=\ttfamily, identifierstyle=\ttfamily, backgroundcolor=, frame=none]
Vi er en projektgruppe på Aalborg Universitet som er ved at udvikle en applikation med madopskrifter. I den forbindelse vil vi gerne vide hvordan ophavsrettighederne omfatter opskrifter.

Det er klart at man ikke må kopiere en beskrivelse af en opskrift fra en kogebog ord for ord, men må man bruge fremgangsmåden og listen af ingredienser til f.eks. en æggekage fra en kogebog?

I den amerikanske "copyright" lov er det ret tydeligt at den ikke beskytter en liste af ingredienser.
Link: http://www.copyright.gov/fls/fl122.html

Da der skal skrives en rapport om projektet, vil vi i den forbindelse høre om vi eventuelt må have lov til at citere dit svar i rapporten?

På forhånd tak,
Nicklas, Jesper, Jacob, Simon, Sam
\end{lstlisting}

\newpage

\section*{Answer from Jens H. Bech}
\begin{lstlisting}[numbers=none, breakindent=0pt, basicstyle=\ttfamily,keywordstyle=\ttfamily, numberstyle=\ttfamily, commentstyle=\ttfamily, stringstyle=\ttfamily, identifierstyle=\ttfamily, backgroundcolor=, frame=none]
From: jb@amtmandstoften.dk
Date: 2014-04-01
Title: RE: Spørgsmål om ophavsret

Hej Nicklas
 
Jeg skal straks erkende, at jeg ikke beskæftiger mig meget med ophavsret. Jeg har fundet nedenstående på "Familieadvokaten.dk":
\end{lstlisting}
\begin{lstlisting}[numbers=none, breakindent=0pt, basicstyle=\ttfamily\itshape,keywordstyle=\ttfamily\itshape, numberstyle=\ttfamily\itshape, commentstyle=\ttfamily\itshape, stringstyle=\ttfamily\itshape, identifierstyle=\ttfamily\itshape, backgroundcolor=, frame=none, xleftmargin=20pt, mathescape]
Kan man have ophavsret til madopskrifter?

Det kan godt være, at det ikke lige falder ind under de mest gængse emner her på siden, men jeg har længe undret mig over, om madopskrifter er underlagt ophavsret.
Kan man f.eks. skrive opskrifter af efter bøger og selv udgive de samme opskrifter - måske bare i en anden sammenhæng?
SVAR.
Madopskrifter - eller kulinariske værker om du vil - er omfattet af ophavsretten (efter lovens § 1), da der er tale om en slags faglitterære værker i skrift.
Du har ikke ret "til at planke" - som det populært kaldes - andres opskrifter i blade og bøger for derefter at udgive dem i bogform.
Men benytter du en offentliggjort opskrift til at udvikle din helt egen opskrift - fx med flere ingredienser og lidt anderledes mængdemål - har du frembragt et nyt kulinarisk værk, som du frit kan offentliggøre i fx en bog.
Med venlig hilsen
Erik Frodelund

§ 1. Den, som frembringer et litterært eller kunstnerisk værk, har ophavsret til værket, hvad enten dette fremtræder som en i skrift eller tale udtrykt skønlitterær eller faglitterær fremstilling, som musikværk eller sceneværk, som filmværk eller fotografisk værk, som værk af billedkunst, bygningskunst eller brugskunst, eller det er kommet til udtryk på anden måde.
Stk. 2. Kort samt tegninger og andre i grafisk eller plastisk form udførte værker af beskrivende art henregnes til litterære værker.
Stk. 3. Værker i form af edb-programmer henregnes til litterære værker.
\end{lstlisting}
\begin{lstlisting}[numbers=none, breakindent=0pt, basicstyle=\ttfamily,keywordstyle=\ttfamily, numberstyle=\ttfamily, commentstyle=\ttfamily, stringstyle=\ttfamily, identifierstyle=\ttfamily, backgroundcolor=, frame=none]
Jeg kan endvidere se, at en dom fra 1983 (UfR 1983515H) slog fast, at en sammensætning af serier af mindre billeder (Mad fra A til Z) ikke blev anset for at være et litterært værk.
 
Håber I kan bruge noget af ovenstående eller skriv igen.
 
Med venlig hilsen
 
Jens H. Bech
\end{lstlisting}

\newpage

\section*{Answer from Jørgen L. Steffesen}
\begin{lstlisting}[numbers=none, breakindent=0pt, basicstyle=\ttfamily,keywordstyle=\ttfamily, numberstyle=\ttfamily, commentstyle=\ttfamily, stringstyle=\ttfamily, identifierstyle=\ttfamily, backgroundcolor=, frame=none]
From: Jørgen Lindhardt Steffesen (Startværkst.dk)
Date: 2014-03-27
Title: RE: Ophavsret på opskrifter

Det er det man kalder et godt spørgsmål. Efter oplysningerne arbejder I reelt på at udgive en applikationsbaseret "kogebog", hvor det formentlig er tanken at man kan downloade en app der indeholder diverse madopskrifter. Så langt så godt. Det lyder som en glimrende idé. Næste skridt er så hvilke opskrifter brugeren kan vælge på app'en, og navnlig hvor de kommer fra. Hvis I eksempelvis kopierer en opskrift fra en af Claus Meyer's bøger om f.eks. ribbensburger - der iøvrigt er rigtig god - får i problemer med copyright. Dels må I ikke kopiere fra bøger uden tilladelse, og dels vil en ordret gengivelse af opskriften være en krænkelse af Meyer's rettigheder. Han anvender i den pågældende burger bl.a. senneps-mayonaisse og æbler, hvormed han adskiller sig fra det der kan kaldes en standard-version. Til gengæld vil en gengivelse af standard-versionen ikke indeholde en krænkelse da det er alment kendt. Så længe I holder Jer til opskrifternes standardindhold og fremgangsmåde krænker I ikke nogen rettigheder.
\end{lstlisting}