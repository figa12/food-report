This chapter describes how the API has been implemented on the server as well as how the search queries are designed and implemented.
\section{API}
\label{sec:com}

The server is implemented as a REST API using GET parameters, this means that whenever a request is made to the server, it is of the following format: $$http://figz.dk/food/lib/ingredients.php?lang=us$$ In the given example the parameter sent is \textit{lang} with the value of \textit{"us"}.

We have created a single file on the server that handles all database connections, this means that whenever a call is made to the server, the parameters are checked and then sent to an instance of the database class. This way we isolate all connection to the database making it easier to maintain and test. The server will then return a \ac{json} formatted object which can easily be deserialised on the mobile application. We have chosen \ac{json} as a format since it is well supported, easy to generate, and deserialisable in both PHP and Android.

\begin{lstlisting}[language=phpstyle, caption=ingredients.php.]
require_once('DB.php');
// Connect to the DB
$DB = new DB();

if (isset($_GET["lang"]))
	$lang = $_GET["lang"];
else
	$lang = "us";

$DB->getIngredients($lang);
\end{lstlisting}%$

\begin{description}
\item[Lines 1-3] Include the database file and create an instance of the \inline{DB} class. The constructor of the \inline{DB} class will automatically create a connection to the database.
\item[Lines 5-8] Check whether the \textit{lang} parameter is set. If the parameter is not set it will be defaulted to \textit{"us"}.
\item[Line 10] The method \inline{getIngredients()} in the \inline{DB} class is called with the given language parameter.
\end{description}

\begin{lstlisting}[language=phpstyle, label=lst:getIngredients, caption=getIngredients() method of DB class.]
public function getIngredients($lang) {
    if ($this->checkLanguage($lang)) {
        $ingredientQuery = $this->con->prepare(
            "SELECT ingredient.id, singular.{$lang} AS singular, plural.{$lang} AS plural
             FROM ingredient
             JOIN string AS singular ON ingredient.singular = singular.id
             JOIN string AS plural ON ingredient.plural = plural.id
             ORDER BY singular.{$lang} ASC");
        $ingredientQuery->execute();
        return $this->echoJson($ingredientQuery);
    }
}
\end{lstlisting}%$

\begin{description}
\item[Line 1] \inline{getIngredients()} takes the language as a parameter.
\item[Line 2] \inline{checkLanguage()} is called to check whether the language is valid. The method simply checks that the string matches a string in an array of valid strings.
\item[Lines 3-8] The SQL query is prepared for execution with the language parameter.
\item[Line 9] The query is executed.
\item[Line 10] The \inline{echoJson()} method is called to print the result of the SQL query.
\end{description}

As seen in \autoref{lst:getIngredients} the method \inline{echoJson()} is called. This method will serialise the result into a \ac{json} formatted object and then print it. An example of result from a request to the server could look like the following:

\begin{lstlisting}[language=json, label=lst:jsonResult, caption=Example result from \colorbox{background}{getIngredients()}.]
[
    {
        "id": 7,
        "singular": "Cointreau",
        "plural": "Cointreau"
    },
    {
        "id": 1,
        "singular": "cream",
        "plural": "cream"
    },
    {
        "id": 2,
        "singular": "dark chocolate",
        "plural": "dark chocolate"
    },
    ...
]
\end{lstlisting}
In \autoref{lst:jsonResult} it is worth noticing that the result is an array of \ac{json} objects where every row that was selected from the database is represented as an object. Since the database contains a lot of ingredients, this example is only a snippet of the total result. The different types of requests the server can take are:

\begin{description}
\item[favourites] This request takes two required parameters and four optional parameters. The two required parameters are: \textit{action} and \textit{hash}. The four optional parameters are: \textit{recipeid}, \textit{limit}, \textit{offset}, and \textit{lang}.

\begin{description}
\item[action] must be set to one of four things: \textit{status}, \textit{add}, \textit{remove}, and \textit{get}. \textit{hash} must be set to the SHA-256 hashed value of the user's email.
Depending on the set value of \textit{action} the following will happen:
\item[status] Requires \textit{recipeid} to be set. This request will return a boolean value indicating whether the recipe is added to the user's list of favourites or not.
\item[add] Requires \textit{recipeid} to be set. This request will add the given recipe to the user's list of favourites .
\item[remove] Requires \textit{recipeid} to be set. This request will remove the given recipe from the user's list of favourites.
\item[get] Requires \textit{limit}, and \textit{offset} to be set. This request will return all recipes from the user's list of favourites.
\end{description}

\item[ingredientsearch] This request takes one required parameter and three optional parameters. The required parameter is \textit{i} and it must be set to an array of ingredient ids. The optional parameters are: \textit{lang}, \textit{limit}, and \textit{offset}. The request will return an array of partial recipes. A partial recipe consists of the recipe id, name, and image name.
\item[ingredients] This request takes one optional parameter: \textit{lang}. If \textit{lang} is not set it will default to ``us''. The request will return an array of all ingredients in the system in the given language.
\item[recipe] This request takes one required parameter and two optional parameters. The required parameter is \textit{id} which must be set to a recipe id. The optional parameters are \textit{lang} and \textit{metric}. This request will return an object containing all information about the given recipe.
\item[textsearch] This request takes one required parameter and three optional parameters. The required parameter is \textit{q} and it must be set to the string of which you want to search for. The optional parameters are \textit{lang}, \textit{limit}, and \textit{offset}. The request will return an array of partial recipes that satisfy the search query.
\end{description}



%%% Local Variables: 
%%% mode: latex
%%% TeX-master: "../master"
%%% End: 
