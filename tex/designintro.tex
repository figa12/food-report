This section describes how we have used user stories as basis for our navigation design. The section also shortly describes our system architecture and shows our early and later stages prototypes.

The reader is expected to be familiar with standard Android components. Some components are briefly explained here:
\begin{description}
\item[Activity] An activity is a single focused thing that the user can do. You can also say that it is a window which is either full-screen or floating. \citep{activity}
\item[Fragment] A fragment is nested in an activity which means its lifecycle is tied to its parent activity. Fragments can be used to build a multi-pane user interface. \citep{fragment}
\item[Action Bar] The action bar is a window feature that identifies your location in the application, and provides user actions and navigation modes. It is located at the top of the screen. \citep{actionbar}
\end{description}

However, we are also referring to activities and fragments as pages, this is because it is irrelevant to the context what kind of implementation is used.