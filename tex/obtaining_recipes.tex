\section{Obtaining Recipes}
An application for finding recipes is not of much value without a sizeable collection of recipes. It would be a major overhead for our project if we had to create all recipes available in the system. This includes a title, description, instructions, and pictures, for each recipe.

If we want to make recipes for the system, we would undoubtedly be inspired by existing recipes. We decided to contact two lawyers to help clarify the Danish copyright law regarding recipes. Their full responses can be seen in \appref{app:copyrightmails}. It became clear that the law is quite vague about this subject. If we change some ingredients in an existing recipe we can legally call it our own.
\begin{quote}
\textit{Recipes - or culinary works if you want - are covered by copyright (in accordance with the law's § 1), as they are a kind of non-fiction works in print.}\par\raggedleft--- \textup{Erik Frodelund}, \textit{Familieadvokaten.dk}
\end{quote}
\begin{quote}

\textit{...are you using a published recipe to develop your very own recipe - for example with more ingredients and a little different quantities - you have created a new culinary work that you are free to publish in for example a book.}\par\raggedleft--- \textup{Erik Frodelund}, \textit{Familieadvokaten.dk}
\end{quote}
We are also allowed to use recipes which are considered to be ``standard-versions''. A ``standard-version'' recipe is vaguely defined as a recipe which is ``generally known''.
\begin{quote}
\textit{As long as you stick to the recipes' standard-content and approach you are not violating any rights.}\par\raggedleft--- \textup{Jørgen Lindhardt Steffesen}, \textit{Startvækst.dk}
\end{quote}
The obvious disadvantages of creating a recipe based on copyrighted material is that we need to write the description, instructions, and obtain pictures of the resulting dish ourselves.

It is also possible to obtain recipes which are licensed by a \emph{Free Cultural Works}\cite{freedomdefined} license. \emph{Free Cultural Works} are works which can be used for free, copied, and modified, for any purpose. This includes all works covered by the Creative Commons\cite{creativecommons} licenses that allow commercial use and modification of the material. The following websites contain recipes of which many are licensed by \emph{Free Cultural Works} licenses:
\begin{itemize}
	\item \url{http://en.wikibooks.org/wiki/Cookbook:Recipes}
	\item \url{http://www.nibbledish.com/}
	\item \url{http://www.opensourcefood.com/}
\end{itemize}

No matter how the recipes are obtained, we are required to adapt every recipe to fit our data structure. While some content can be crawled from the sources, different formatting between the content creators makes it hard, if not impossible, to automate the adaptation of the content to our data structure.