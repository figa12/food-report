\section{Obtaining recipes}
An application for finding recipes is not of much value without a sizeable collection of recipes. It would be a major overhead of the project if we had to create all recipes available in the system. This includes a title, description, instructions, and pictures, for each recipe.

\todo{Ophavsret svar fra startvækst} ... The obvious disadvantages of creating a recipe based on copyrighted material is that we still need to write the description, instructions, and obtain pictures of the resulting dish ourselves.

It is also possible to obtain recipes which are licensed by a \emph{Free Cultural Works}\cite{freedomdefined} license. \emph{Free Cultural Works} are works which can be used for free, copied, and modified, for any purpose. This includes all works covered by the Creative Commons\cite{creativecommons} licenses that allow commercial use and modification of the material. The following websites contain recipes of which many are licensed by \emph{Free Cultural Works} licenses:
\begin{itemize}
	\item \url{http://en.wikibooks.org/wiki/Cookbook:Recipes}
	\item \url{http://www.nibbledish.com/}
	\item \url{http://www.opensourcefood.com/}
\end{itemize}

No matter how the recipes are obtained, we are required to adapt every recipe to fit our data structure. While some content can be crawled from the sources, different formatting between the content creators makes it hard, if not impossible, to automate the adaptation of the content to our data structure.