The focus of this project was to make an application that would take advantage of the mobile platform and provide the users with relevant recipes based on specific ingredients. Furthermore we wanted to support any user defined restrictions, like allergies.
This is defined in our problem definition from \chapref{chap:intro}:
\begin{quote}
How can we take advantage of the mobile platform, in order to provide the user with relevant recipes based on specific ingredients, and taking any user defined restrictions, like allergies, into consideration.
\end{quote}
In order to make the application easy for the users to use, a problem that had to be solved was how the users should navigate the application. We decided to use the navigation drawer and display it the first time the application is opened. We chose the navigation drawer because it is recommended to use when the application has more than three top level views in order to give the user an overview of the application.

The application itself consists of three pages, a page to search by ingredient, a page to search for recipes using free-text, and a favourite page. 

When the user wants to search for recipes using ingredients, they click the search field, ingredients and a word cloud appear filled with already predefined ingredient suggestions. The user is able to click these and add them to the search, they are also able to search for ingredients using text. When the user types in the search field they are given a list of suggestions based on the letters they have already typed. The user can either click an ingredient in the list to add it to the search or click the enter button on the keyboard to add the first suggestion in the list to the search. When the user is done entering ingredients they can perform a search by clicking the enter button while having an empty search field. This closes the word cloud and searches for recipes based on the ingredients that were selected. 

The user is given a list of recipes that either include all or some of the ingredients that they entered. The recipes are prioritised accordingly to our precedence function described in \secref{sec:design_search}. The user can click a recipe from the list to open a page showing the full recipe. The users are able to favourite recipes by clicking the star in the top right corner. In order to favourite recipes they must be signed in through Google+. As long as they are signed in the users can always access the recipes through the favourite page. Through the favourite list they can also remove favourites by long clicking and click "remove" when prompted. They can also remove a favourite recipe by clicking the star in a favourited recipe.

The user is also able to search for recipes using free-text using wildcards such as *, + and "". In order to receive a result from the free-text search, the input text has to match either parts of the recipe name or part of the description of a recipe. A match on the recipe title is prioritised over a match on the description, and a match on both title and description is prioritised over a match on the title.

Due to time constraints, we have not managed to fulfil all of our initial requirements as listed in \secref{sec:requirement}. One of the major features that is lacking is the shopping list. It proved more complicated and time consuming to implement than first anticipated due to too much UI design on the application.

The intended purpose of the word cloud was to provide the user with relevant suggestions based on the ingredients they already entered to the search, however at the moment the suggestions provided are static and does not change. This basic implementation was a result of time constraints and how complicated it proved to implement something that gives relevant results. 

As the project progressed, we chose to focus on developing a working application with a few core features which meant that features such as search filters, sharing, persistency, unit conversion, and additional languages where deprioritised. 

Based on the tests performed in \chapref{chap:tests}, we can guarantee\todo{nej} that the server returns a correct and valid \ac{json} format to the mobile application. Our black box test shows that the mobile application's functionalities works. We have not made any graphical user interface tests to ensure the design of our mobile application and website. We have managed to create an application which provides the users with the ability to search for recipes based on ingredients or using free-text.