%authentication
\section{Authentication}
We have not implemented our own sign in option, since it would require us to handle the user's passwords and other personal information.
By using a social media sign in In our application as a way of identify the user, we remove the security measures to the specific social media.
There are many different social media sign in options; Google+, Facebook, Twitter etc., but the sign in we chosen to implement first is Google+. 
Google recommends using the Google+ sign in for Android applications\todo{kilde?}, since the developer is able to gather information about the user through their Google+ account. The other sign in option could later be implemented.

\subsection{Google+}
It makes sense to use the Google+ sign in since it is highly recommended by Google to have a Google account when using an Android device, since the user would limit their experience with an Android device, if they do not have a Google account. The Google+ authentication is natively supported by Android and can be achieved seamlessly. 


