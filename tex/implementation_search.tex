\section{Search}
\todo{intro}\todo{dette stykke skal nok stå et andet sted}
\textbf{It should be noted that variables are only inserted directly in the SQL queries for clarification.} Inserting user specified variables directly in an SQL query will make the query vulnerable to SQL injections. In the actual implementation, the SQL queries are protected using PHP's \textit{prepared statements}. In prepared statements, the query and variables are separated to make make sure that the values of the variables are not executed on the \ac{rdbms}. Prepared statements can also be used to execute a query repeatedly with different values, without sending the entire query to the \ac{rdbms} every time.

Since SQL is a declarative language, each individual SQL query's clauses will not be explained sequentially but in an order that is easier to understand.



\section{Free-text search}
Free-text search can be used to search for parts of the recipe name or words in the recipe description. As defined in \secref{sec:design_database}, we use MariaDB as our \ac{rdbms}. MariaDB has support for full-text search which makes it simple to implement free-text search. MariaDB also has support for a special full-text search called ``Boolean search''.\cite{mariafulltext} In ``Boolean search'', the user is able to use special operators like  +, -, "", and *  for marking words as mandatory, discarding rows containing specific word, matching phrases, and inserting wildcards, among other things. Unlike the default search mode, ``Natural Language'', which returns a score for each row, the ``Boolean search'' is that this query only outputs ``true'' or ``false'', to indicate whether a match was found. We want to be able to order the result based on how well the text in the recipes match the query. In particular, a match in a title of a recipe should have higher precedence in the result than a match in a description of a recipe.
\autoref{lst:freetext} shows the SQL query we use for free-text search.

\begin{lstlisting}[language=SQL, morekeywords={AGAINST,OFFSET,BOOLEAN,MODE}, float=h, label={lst:freetext}, caption={Free-text search.}]
SELECT recipe.id, name.{$lang} AS name, image,
  (MATCH (name.{$lang}) AGAINST ('{$str}')) AS title_score,
  (MATCH (description.{$lang}) AGAINST ('{$str}')) AS desc_score
FROM recipe
JOIN string AS name ON name.id = recipe.name
JOIN string AS description ON description.id = recipe.description
WHERE MATCH (name.{$lang}, description.{$lang})
  AGAINST ('{$str}*' IN BOOLEAN MODE)
ORDER BY title_score DESC, desc_score DESC
LIMIT {$limit} OFFSET {$offset}
\end{lstlisting}
\begin{description}
\item[Lines 4-6] The \inline{recipe} table is joined with the \inline{string} table two times to get both the title and description of each recipe.
\item[Lines 7-8] Keep only recipes which have a match in \inline{BOOLEAN MODE}. Note the asterisk, `*', which is appended to the user defined search text. This allow the user to enter e.g. ``choco'' and get a match on e.g. ``Chocolate cake''. We would like to add a prefixed asterisk as well to allow users to search for e.g. ``berry'' and get results like ``Strawberry Pie''. Unfortunately it is not possible due to limitations in MariaDB.\cite{mariafulltext}
\item[Lines 2-3] The scores for when the user defined text is matched to the recipes' titles and descriptions are calculated using the default full text search mode: \inline{NATURAL LANGUAGE MODE}. These values are stored in respectively \inline{title_score} and \inline{desc_score}. \todo{so bad}
\item[Line 9] The result is ordered using the calculated scores with \inline{title_score} having a higher precedence than \inline{desc_score}.
\item[Line 10] The result is limited and offset according to the applications needs i.e. how far the user has scrolled through the result.
\item[Line 1] The ID, name, image, and scores of the recipes are returned from the query and then later send to the application.
\end{description}



\section{Search by ingredients}
The application also allows the user to search for recipes by entering a number of ingredients. The search algorithm should then find recipes which contain one or more of the ingredients specified by the user. The search algorithm should then order the results according to the precedence function described in \secref{sec:design_search}.

The source code for the search algorithm, excluding SQL queries, is available in \appref{app:search}. The general structure of the algorithm is as follows:
\begin{enumerate}
\item The input parameters are validated.
\item A search query is sent to the DBMS together with the input parameters. The result is an ordered list of recipes.
\item For each recipe we find the exchangeables that are not \textit{covered} by the user specified ingredients. Exchangeables allow us to specify which ingredients in a recipe can be replaced by other ingredients. An example could be a recipe where either minced beef or minced pork could be used. If the user has specified either one in the search query we consider the exchangeable to be \textit{covered} i.e. the user does not need to buy minced pork if he searched for ``minced beef''.
\item Using the list of exchangeables that are not covered, we for each recipe create a list containing the ingredients they also need in order to follow the recipe. This list excludes the ingredients the user specified for the search.
\item The result is returned to the user.
\end{enumerate}

The search query mentioned above clearly does most of the work in the search algorithm by returning an ordered list of recipes related to the ingredients specified by the user. The query itself has the following structure:
\begin{enumerate}
\item Calculate how many of the user specified ingredients exists in each recipe.
\item Calculate how many not covered exchangeables exists in each recipe.
\item Retrieve the name, description, and image for each recipe and order them according to the precedence function described in \secref{sec:design_search}.
\end{enumerate}

Each part of the search query will be described in detail on the following pages.

\subsection*{\$quantityQuery}
\autoref{lst:quantity} shows the SQL query \inline{$quantityQuery}, that returns every quantity that contains one of the ingredients specified by the user.
\begin{lstlisting}[language=SQL, float=h, label={lst:quantity}, caption={\$quantityQuery, return quantities that match the search.}]
SELECT DISTINCT exchange_ingredient_id, ingredient_id
FROM quantity
WHERE FIND_IN_SET(ingredient_id, '{$commaSeparatedIngredients}')
\end{lstlisting}
\begin{description}
\item[Lines 2-3] Get the quantities which are made of one of the ingredients defined by the user\todo{Kan vi finde en bedre beskrivelse}. \inline{FIND_IN_SET} allow us to check a comma separated string instead of a tuple required by the \inline{IN} operator. It would not be possible to provide a tuple to the \inline{IN} operator because PHP's prepared statements does not have support for a variable number of arguments.
\item[Line 1] The IDs of the found pairs of exchangeables and ingredients are returned.
\end{description}



\subsection*{\$ingredientQuery}
\autoref{lst:ingredient} shows the SQL query \inline{$ingredientQuery}, that returns the number of times the individual ingredients match in each recipe.
\begin{lstlisting}[language=SQL, float=h, label={lst:ingredient}, caption={\$ingredientQuery, returns the number of times the individual ingredients appear in each recipe.}]
SELECT recipe_id, ingredient_id,
  SUM(mandatory) AS mandatory_matching,
  SUM(NOT mandatory) AS optional_matching
FROM ( {$quantityQuery} ) AS temp
JOIN exchangeable_ingredients
  ON temp.exchange_ingredient_id = exchangeable_ingredients.id
GROUP BY recipe_id, ingredient_id
\end{lstlisting}
\begin{description}
\item[Lines 4-6] The result of the \inline{$quantityQuery} is joined with the \inline{exchangeable_ingredients} table.
\item[Line 7] The result of the join is then grouped by \inline{recipe_id} and \inline{ingredient_id}.
\item[Lines 1-3] The IDs of the pairs of recipes and ingredients are returned together with the sum of the \inline{mandatory} column in \inline{exchangeable_ingredients}. Since the \inline{mandatory} column consists of boolean values we will simply get the number of times this ingredient appear in the recipe as mandatory. Likewise, we get the number of times this ingredient appear in the recipe as optional.
\end{description}



\subsection*{\$matchingQuery}
\autoref{lst:matching} shows the SQL query \inline{$matchingQuery}, that returns the number of matching ingredients for each recipe.
\begin{lstlisting}[language=SQL, morekeywords={OFFSET}, float=h, label={lst:matching}, caption={\$matchingQuery, returns the matching ingredients count for each recipe.}]
SELECT recipe_id,
  COUNT(NULLIF(mandatory_matching,0)) AS mandatory_matching,
  COUNT(NULLIF(optional_matching,0)) AS optional_matching
FROM ( {$ingredientQuery} ) AS temp
GROUP BY recipe_id
LIMIT {$limit} OFFSET {$offset}
\end{lstlisting}
\begin{description}
\item[Lines 4-5] The result of \inline{$ingredientQuery} is grouped by the \inline{recipe_id}.
\item[Line 6] The result is limited and offset according to the applications needs i.e. how far the user has scrolled through the result.
\item[Lines 1-3] The ID of each recipe, and the number of matching mandatory and optional ingredients are returned. To calculate this, we first replace the matching count with \inline{NULL} if the count is 0. Since \inline{NULL} values are ignored by the \inline{COUNT} function, we will get the number of unique ingredient matches.
\end{description}



\subsection*{\$exchangeQuery}
\autoref{lst:exchange} shows the SQL query \inline{$exchangeQuery}, that returns for each recipe, the number of times the individual exchangeable contains an ingredient that match a user specified ingredient. The query returns both the number of times a mandatory exchangeable has a match as well as the number of times an optional exchangeable has a match. An exchangeable can obviously not have both mandatory and optional matches at the same time since it is the exchangeable that defines whether its containing ingredients are mandatory or not.
\begin{lstlisting}[language=SQL, float=h, label={lst:exchange}, caption={\$exchangeQuery, returns the number of times the individual exchangeable appear in each recipe.}]
SELECT recipe_id, exchange_ingredient_id,
  SUM(mandatory) AS mandatory_cover,
  SUM(NOT mandatory) AS optional_cover
FROM ( {$quantityQuery} ) AS temp
JOIN exchangeable_ingredients
  ON temp.exchange_ingredient_id = exchangeable_ingredients.id
WHERE temp.ingredient_id NOT IN ( {$ignoredIngredients} )
GROUP BY recipe_id, exchange_ingredient_id
\end{lstlisting}
\begin{description}
\item[Lines 4-6] The result of the \inline{$quantityQuery} is joined with the \inline{exchangeable_ingredients} table.
\item[Line 7] All rows containing ignored ingredients are discarded. Ignored ingredients are ingredient that we assume people have i.e. water, salt, pepper.
\item[Line 8] The result is then grouped by \inline{recipe_id} and \inline{exchange_ingredient_id}.
\item[Lines 1-3] The IDs of the pairs of recipes and exchangeables are returned together with the sum of the \inline{mandatory} column in \inline{exchangeable_ingredients}. Since the \inline{mandatory} column consists of boolean values we will get the number of matching non-ignored ingredients of this exchangeable when the exchangeable is mandatory. Likewise, we get the number of matching non-ignored ingredients of this exchangeable when the exchangeable is optional.
\end{description}



\subsection*{\$coverQuery}
\autoref{lst:cover} shows the SQL query \inline{$coverQuery}, that returns for each recipe, the number of exchangeables that are covered by matching ingredients. An exchangeable is covered when at least one ingredient in the exchangeable match one of the ingredients specified by the user. The query returns both the number of covered mandatory exchangeables and optional exchangeables.
\begin{lstlisting}[language=SQL, morekeywords={NULLIF,OFFSET}, float=h, label={lst:cover}, caption={\$coverQuery, returns the exchangeable cover counts.}]
SELECT recipe_id,
  COUNT(NULLIF(mandatory_cover,0)) AS mandatory_cover,
  COUNT(NULLIF(optional_cover,0)) AS optional_cover
FROM ( {$exchangeQuery} ) AS temp
GROUP BY recipe_id
LIMIT {$limit} OFFSET {$offset}
\end{lstlisting}
\begin{description}
\item[Lines 4-5] The result of \inline{$exchangeQuery} is grouped by the \inline{recipe_id}.
\item[Line 6] The result is limited and offset according to the applications needs i.e. how far the user has scrolled through the result.
\item[Lines 1-3] The ID of each recipe, and the number of covered mandatory and optional exchangeables are returned. To calculate this, we first replace the cover count with \inline{NULL} if the count is 0. Since \inline{NULL} values are ignored by the \inline{COUNT} function, we will get the number of uniquely covered exchangeables.
\end{description}



\subsection*{\$countQuery}
\autoref{lst:count} shows the SQL query \inline{$countQuery}, that returns the total number of mandatory and optional exchangeables for each recipe.
\begin{lstlisting}[language=SQL, float=h, label={lst:count}, caption={\$countQuery, returns the total number of mandatory and optional exchangeables for each recipe.}]
SELECT recipe_id, SUM(mandatory) AS mandatory_count,
  SUM(NOT mandatory) AS optional_count
FROM exchangeable_ingredients
JOIN quantity
  ON quantity.exchange_ingredient_id = exchangeable_ingredients.id
WHERE quantity.ingredient_id NOT IN ( {$ignoredIngredients} )
GROUP BY recipe_id
\end{lstlisting}
\begin{description}
\item[Lines 3-5] The \inline{quantity} table is joined with the \inline{exchangeable} table.
\item[Line 6] The rows which consists of an ignored ingredient are discarded.
\item[Line 7] The result is grouped by \inline{recipe_id}.
\item[Lines 1-2] The ID of each recipe is returned together with the number of mandatory and optional exchangeables for that recipe.
\end{description}



\subsection*{\$lackingQuery}
\autoref{lst:lacking} shows the SQL query \inline{$lackingQuery}, that returns the number of lacking ingredients for each recipe.
\begin{lstlisting}[language=SQL, morekeywords={IFNULL}, float=h, label={lst:lacking}, caption={\$lackingQuery, returns the number of lacking ingredients for each recipe.}]
SELECT temp2.recipe_id,
  (mandatory_count-IFNULL(mandatory_cover,0)) AS mandatory_lacking,
  (optional_count-IFNULL(optional_cover,0)) AS optional_lacking
FROM ( {$coverQuery} ) AS temp1
RIGHT JOIN ( {$countQuery} ) AS temp2
  ON temp1.recipe_id = temp2.recipe_id
\end{lstlisting}
\begin{description}
\item[Lines 4-6] The result of \inline{$coverQuery} is joined with the result of \inline{$countQuery}. We use \inline{RIGHT JOIN} to ensure we do not discard recipes where a match exists. \todo{Vi kan måske undgå det hvis vi bruger left join i sortingQuery}
\item[Lines 1-3] The ID of each recipe, and the number of lacking mandatory and optional exchangeables are returned. To calculate this, we first replace the cover count with 0 if the count is \inline{NULL} due to the \inline{RIGHT JOIN} in \inline{$lackingQuery}. We then substract the mandatory cover count from the total number of mandatory exchangeables to get the number of lacking exchangeables. We do the same to get the number of lacking optional exchangeables.
\end{description}



\subsection*{\$sortingQuery}
\autoref{lst:sorting} shows the SQL query \inline{$sortingQuery}, that returns a list of recipes sorted according to the precedence function defined in \secref{sec:design_search}.
\begin{lstlisting}[language=SQL, float=h, label={lst:sorting}, caption={\$sortingQuery, combine and sort.}]
SELECT recipe.id, name.{$lang} AS name, image
FROM ( {$matchingQuery} ) AS temp1
JOIN ( {$lackingQuery} ) AS temp2 ON temp1.recipe_id = temp2.recipe_id
JOIN recipe ON recipe.id = temp1.recipe_id
JOIN string AS name ON name.id = recipe.name
ORDER BY mandatory_matching DESC , mandatory_lacking ASC ,
  optional_matching DESC , optional_lacking ASC
\end{lstlisting}
\begin{description}
\item[Lines 2-3] The result of \inline{$matchingQuery} is joined with the result of \inline{$lackingQuery}.
\item[Lines 4-5] The rows are then joined with the \inline{recipe} table and the \inline{string} table. 
\item[Lines 6-7] The result is ordered as defined in \secref{sec:design_search}. In descending order of precedence: most mandatory ingredients matching, least mandatory exchangeables lacking, most optional ingredients matching, least optional exchangeables lacking.
\item[Line 1] The ID, name, and image of the recipes are returned from the query and then later send to the application.
\end{description}





\subsection*{Get needed ingredients}
We now have all the recipes returned by the search query but we also want to provide the user with a list of ingredients they also need in order to follow the recipes, i.e. the ingredients in the exchangeables that are not covered by the ingredients specified by the user. Those ingredients are found by first finding the exchangeables that are not covered using the query defined in \autoref{lst:getNotCovered}.
\begin{lstlisting}[language=SQL, morekeywords={FIND_IN_SET}, float=h, label={lst:getNotCovered}, caption={Query to find the exchangeables that are not covered.}]
SELECT id
FROM exchangeable_ingredients
WHERE exchangeable_ingredients.recipe_id = {$recipeId}
AND mandatory = 1
AND NOT EXISTS (
    SELECT * 
    FROM quantity
    WHERE exchangeable_ingredients.id = quantity.exchange_ingredient_id
    AND (
    	FIND_IN_SET(ingredient_id, {$ingredients})
    	OR FIND_IN_SET(ingredient_id, '{$ignoredIngredients}' )
    )
)
\end{lstlisting}
\begin{description}
\item[Line 2-3] The exchangeables which are a part of a specified recipe are retrieved.
\item[Line 4] Optional exchangeables are discarded.
\item[Lines 5-13] Discard exchangeables which are covered by the ingredients specified by the user.
	\begin{description}
	\item[Lines 7-8] Retrieve quantities which are a part of a given exchangeable.
	\item[Line 9-12] Discard quantities which does not contain one of the ingredients specified by the user or an ignored ingredient.
	\item[Line 6] If the query returns any rows it means that the exchangeable is covered by any f the ingredients specified by the user.
	\end{description}
\item[Line 1] The IDs of the resulting exchangeables are returned. These exchangeables are not covered by any user specified ingredients.
\end{description}


We now have the exchangeables for each recipe that are not covered by an ingredient specified by the user or an ignored ingredient. We then simply find the ingredients that are a part of those exchangeables using the query in \autoref{lst:getIngredientsQ}.
\begin{lstlisting}[language=SQL, morekeywords={FIND_IN_SET}, float=h, label={lst:getIngredientsQ}, caption={Query to get the ingredient of an exchangeable, excluding ignored ingredients.}]
SELECT DISTINCT ingredient_id
FROM quantity
WHERE exchange_ingredient_id = ?
\end{lstlisting}
\begin{description}
\item[Lines 2-3] The quantities which are a part of a specified exchangeable are retrieved.
\item[Line 1] The ingredient IDs of the resulting quantities are returned.
\end{description}