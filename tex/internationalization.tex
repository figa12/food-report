\section{Internationalization}
We want our system to be internationalized such that it is ready for localization.

Apart from having to provide the ability to add multiple languages, internationalization also cause other difficulties for some cultures:

\begin{description}
  \item[Reading direction] While most languages are written left-to-right top-to-bottom, some languages, like Japanese, are written top-to-bottom right-to-left.
  \item[Special characters] Some languages require characters which are not present in ASCII, e.g. ``æøå'' in Danish.
  \item[Text length] Text may be significantly longer when translated to other languages. An example is text displaying how many times an item has been viewed, e.g. ``5 views''. Translated to Italian that is written ``5 visualizzazioni''\cite{wordlength}.
  \item[Number formatting] While most of the world is using ``dot'' as the decimal mark, many cultures are using other decimal marks, like ``comma''.
  \item[Units of measure] Most of the world is using metric(SI) units for measures; however, USA, Liberia, and Burma have not yet adopted the metric system\cite{unitsfactbook}.
  \item[Prohibited foods] Some types of food may be prohibited or tabooed in certain cultures and religions.
\end{description}