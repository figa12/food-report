\section{Unit test}
The mobile application is designed to be a thin client that sends and recieves messages to and from the server. Because of this we have chosen to focus our testing on the server, if the server has a 100\% code coverage and all tests are satisfied, we only need to ensure that the mobile application sends a correct message, and we can guarantee that the result is correct.
The server layer consists of a class that handles all connection to the database, the goal is to achieve 100\% method coverage on this class.

To avoid corrupting the data that is running on production we have made a script that makes an exact copy of the database to perform all of the testing on. The copying of the database could prove to be a bottleneck if the application were to get very big, but at the in current state it only takes a few seconds.

The unit tests are run using ``PHPUnit'' version 3.6.10 by Sebastian Bergmann,\todo{CITE THIS TOOL http://phpunit.de/} furthermore mutation tests are run using ``judgedim/mutagenesis''\todo{CITE THIS TOOL https://github.com/judgedim/mutagenesis}.

To demonstrate how we have performed the unit tests, the tests \lstinline|testSetRecipe()| and \lstinline|testGetRecipe()| have been included. These tests ensure that when a new recipe is written to the database, the exact same recipe can be retrieved again with no missing data.

\begin{lstlisting}
    /**
     * @depends testSetUnit
     * @depends testSetIngredient
     * @covers DB::setRecipe
     */
    public function testSetRecipe($unit, $ingredient) {
        $url = 'http://localhost/food/lib/recipebackend.php';

        //what post fields?
        $fields = array(
            'test'=>"true",
            'recipeName'=>$this->recipeName,
            'recipeDesc'=>$this->recipeDesc,
            'group1'=>$this->groupName,
            'group1_exchange1'=>"",
            'group1_exchange1_mandatory'=>$this->firstExchangeMandatory,
            'group1_exchange1_ingredient1'=>$ingredient,
            'group1_exchange1_ingredient1_amount'=>$this->firstIngAmount,
            'group1_exchange1_ingredient1_unit'=>$unit,
            'step1'=>$this->firstStepDesc,
            'step2'=>$this->secondStepDesc
            );
        
        $result = $this->sendCurl($url, $fields);
        $this->assertNotNull($result);
            
        $resultRecipeId = (int)strip_tags($result);
        $this->assertNotNull($resultRecipeId);
        
        return $resultRecipeId;
      }
\end{lstlisting}

bla bla bla

\begin{lstlisting}
    /**
     * @depends testSetRecipe
     */
    public function testGetRecipe($recipeId) {
        $jsonResult = self::$DB->getRecipe($recipeId, $this->language, $this->metric);
        $recipe = json_decode($jsonResult);

        //Check that the recipe is equal to the one we inserted.
        $this->assertEquals($recipeId, $recipe->id);
        $this->assertEquals($this->recipeName, $recipe->name);
        $this->assertEquals($this->recipeDesc, $recipe->desc);
        $this->assertEquals($this->recipeImage, $recipe->image);
        $this->assertEquals($this->groupName, $recipe->groups[0]->name);
        $this->assertNotNull($recipe->groups[0]->order);
        $this->assertEquals($this->ingredientName, $recipe->groups[0]->exchanges[0]->ingredients[0]->name);
        $this->assertEquals($this->firstIngAmount, $recipe->groups[0]->exchanges[0]->ingredients[0]->amount);
        $this->assertEquals($this->metricName, $recipe->groups[0]->exchanges[0]->ingredients[0]->unit);

        $this->assertEquals(((bool)$this->firstExchangeMandatory), $recipe->groups[0]->exchanges[0]->mandatory);
    }
\end{lstlisting}


%%% Local Variables: 
%%% mode: latex
%%% TeX-master: "../master"
%%% End: 
