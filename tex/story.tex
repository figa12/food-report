\section{Story, Julie}

\newcommand{\appname}{\textbf{Flappy Cook (?)}\todo{Insert actual application name}}

Julie is in the supermarket, she does not know what to have for dinner later. She finds minced beef on sale, but she does not know what to do with it. She opens our application \appname. She is presented with an activity that says "click above to search for ingredients". She clicks the button, and then a list with categories of ingredients appears. She clicks the category called "meat", and then a word cloud appears with many different kinds of meat. In the word cloud she finds minced beef and clicks it. She now dismisses the ingredient search, and the application now presents Julie with recipes which uses minced beef. Some of the top results include Spaghetti Bolognese and Lasagne. Having not thought about this before, Julie wants to have Spaghetti Bolognese for dinner. She opens the recipe where she can see all the ingredients needed. Remembering which of the ingredients she has at home, she then adds the ingredients she needs to her shopping list. Then she favourites the recipe to easily find it later.

\section{Story, Bob}
Bob is at home, he does not know what to have for dinner later, he opens our application \appname to figure it out. Bob looks in the fridge because there might be something to use. In the fridge he finds minced meat which has a best before date set to today, so he decides that it should be used today. In the application he finds minced meat and clicks it to indicate he wants to use the ingredient. He decides he wants Spaghetti Bolognese for dinner later, but he discovers that he is missing tomato purée. He adds tomato purée to his shopping list and favourites the recipe. His wife Julie has not arrived at home yet, so instead of buying tomato purée himself, he decides to share his shopping list with Julie.
