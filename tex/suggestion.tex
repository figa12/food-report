\section{Search Suggestion}
We want the users to be able to type in the ingredients they have in the refrigerator, however it can be quite tedious to type in the full ingredient names, so we want to aid in the users by providing them with search suggestions as they type that they can press and add to the search. We decided that the ingredients that the user can use for searching has to be included in at least one of the recipes in the system. The reason for this was that we did not want the user to be able to search for random text strings, or to get an empty search because none of their ingredients matched those in the recipes. 

When the users opens the application, all the recipes are loaded into a list, this list of ingredients is then used for the search suggestions. Each time the user types in a letter, the input string is run through a regex trying to match the input string to an ingredient name in the list, the matching ingredients is added to a list and shown on the screen as a search suggestion. This can be seen in \autoref{lst:search}.

\begin{lstlisting}[language=java, label=lst:search, caption={Search suggestions}]
Pattern p = Pattern.compile("(^|\\s)" + query);

for (Ingredient ingredient : allIngredients) {
    Matcher matcher = p.matcher(ingredient.getSingular().toLowerCase());

    if (matcher.find()) {
        suggestionName.add(ingredient.getSingular());
    }
}
\end{lstlisting}

\begin{description}
\item[Line 1] \inline{Pattern.Compile()} returns a compiled form of the regular expression. The expression matches the start of a string or a space and then the input itself. 
\item[Line 3-4] We loop through the list of ingredients to see if the regex matches any of the ingredient names.
\item[Line 6-7] Each, if any, suggestion that is found is added to the suggestion list.
\end{description}

\begin{figure}[H]
\begin{minipage}[b]{0.5\columnwidth}
\centering
\includegraphics[width=0.7\columnwidth]{img/screenshots/searchSuggestion1.png}
\caption{Suggestions with a ``t''\label{fig:suggestt}}
\end{minipage}
\hspace{0.5cm}
\begin{minipage}[b]{0.5\columnwidth}
\centering
\includegraphics[width=0.7\columnwidth]{img/screenshots/searchSuggestion2.png}
\caption{Suggestions with ``to''\label{fig:suggestto}}
\end{minipage}
\end{figure}

\autoref{fig:suggestt} and \autoref{fig:suggestto} shows two examples of how this works. When the user types in a ``t'' they get all ingredients where a part of the string starts with a ``t'', if the users continues and types an ``o'', the suggestions are updated and now only show the words where parts of the string starts with ``to''. 

The user can add ingredients to the word cloud in two ways, they can either type parts of the name as shown in \autoref{fig:suggestt} and \autoref{fig:suggestto} and press the correct suggestion in the list of suggestions or they can type parts, or the full name, and press the enter button on the keyboard. Doing this will add the first ingredient in the suggestion list to the search.