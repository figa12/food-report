During the early stages of the project we began searching for existing solutions to draw inspiration to the design and implementation phases. All members of the group have previously had experience using different recipe sites on the Internet, and we  think that the existing solutions lack functionality. We will in this section review and discuss the existing solutions and draw some good practices from these.  

\section{Supercook}
Supercook\cite{supercook} is a website with focus on searching for recipes by their ingredients. The user is limited to enter a set of ingredients which is defined by Supercook. When adding ingredients to a search the user is aided by autocompletion and a word cloud with ingredients. The word cloud changes according to the type of ingredients you have entered, e.g. if you enter ``vanilla'' the word cloud will change to contain typical cake ingredients. A search on Supercook will result in a prioritised list of links to recipes from several online cookbooks.

By evaluating the search results of Supercook we have deduced what we believe are the steps used to generate the search results:
\begin{enumerate}
	\item Remove recipes without matching ingredients.
	\item Sort by most matching ingredients.
	\item Sort by least missing ingredients.
\end{enumerate}
We have also noticed some quirks in the way it counts matching ingredients. It seems like the search algorithm counts the number of occurences of the entered ingredients in the recipes. When entering ``bell pepper'' and ``olive oil'', recipes which contain both red, green, and yellow pepper are prioritised higher than recipes which actually contains both of the entered ingredients. It also appear as if ingredients in the recipes are mapped to the recipes defined by Supercook. An example of that is that ``flour'' appear to be mapped to ``flour'', ``wheat'', ``oat'', etc. This also appear to have the shortcoming that recipes containing fx ``wheat flour'' are prioritised as if they have two matching ingredient when the user entered ``flour''.

When making a search containing ``eggs'', ``garlic'', ``ground beef'', and ``onion'', we get a result which showcases a drawback of the sorting done by Supercook. The first 94 recipes are simple recipes using only few of the entered ingredients, including 24 recipes on how to boil an egg. Recipe number 95 is a recipe for meatballs and is using all entered ingredients but it also needs ``bread crumps''. The reason why the first 94 recipes are prioritised highest is because they do not need any other ingredients than the ingredients entered.
\todo{Add pictures of the search-bar}


\section{Allthecooks}
Allthecooks is one of the most downloaded \cite{allthecooks-googleplay} Android application that is related to ``recipe''. The application has a great design, it is intuitive and easy to navigate. The detailed display of a recipe is especially well implemented, the user has everything on a single activity, and can easily see the needed ingredients and the required steps to cook the recipe. The user can also by a press of a button calculate between different measuring units, or add the selected ingredient to a shopping list. Screenshot of the application is shown on Figures \ref{fig:allthecooks-menu}, \ref{fig:allthecooks-detail1}, \ref{fig:allthecooks-detail2}, \ref{fig:allthecooks-detail3}. The application is very elegant in its design, the search is free-text based, which means that opposite the Supercook web application it is hard to find recipes with multiple ingredients.
\twofigs{screenshots/menu.png}{Menu of Allthecooks}{fig:allthecooks-menu}{screenshots/rainbowcake-1.png}{Detail display of a recipe}{fig:allthecooks-detail1}
\twofigs{screenshots/rainbowcake-2.png}{Buttons for different features}{fig:allthecooks-detail2}{screenshots/rainbowcake-3.png}{Directions for the recipe}{fig:allthecooks-detail3}

\section{BigOven}
BigOven is also one of the most downloaded \cite{bigoven-googleplay} Android application that is related to ``recipe''. The application's navigation and design does not follow the Android guidelines\cite{guidelines-appstructure} and therefore the application's navigation can be quite confusing to use for an Android user. \todo{insert some pictures of the start page.} 

\section{Review of existing solution}
After a small preview of the already existing solutions, of both web and android applications. We believe that we can create an application with improved features, such as a better search algorithms, easy navigation, and an elegant design. We like the idea of being able to search for recipes by ingredients, but we wont exclude the possibility to also search directly for recipes. We will also provide the user with different input styles; text-type-input(free text for recipe search and ingredient restricted), and tile-input. Tile input is basically the same as ingredient restricted input, but instead of typing the name of each ingredient the user selects the ingredient by use of tiles. The ingredient are divided into categories and maybe even subcategories, to make the navigation more sensible, otherwise the user would be presented with a too many tiles and the screen would be cluttered.
\begin{table}[H]
\centering
\begin{tabular}{|l|l|l|l|}
\hline
 & \textbf{Supercook} & \textbf{Allthecooks} & \textbf{BigOven} \\
\hline
\textbf{Platform} & Website & Android/Website & Android/Website \\
\hline
\textbf{Cost} & Free & Free & Free \& Paid \\
\hline
\textbf{Recipe search} & No & Yes & Yes  \\
\hline
\textbf{Ingredient search} & Yes & No & Yes \\
\hline
\textbf{Recipe origin} & Web crawl & User defined & User defined \\
\hline
\textbf{Shopping list} & Yes & Yes & Yes \\
\hline
\textbf{Menu planner} & No & Yes & Yes \\
\hline
\textbf{Saving recipes} & Yes & Yes & Yes \\
\hline
\textbf{Allergic Filter} & Yes & Yes & Yes \\
\hline
\textbf{Sharing of shopping list} & No & E-mail/SMS & E-mail \\
\hline
\textbf{Sharing of recipe} & No & App/SMS/E-mail & App/E-mail \\
\hline
\end{tabular}
\caption{Application comparison}
\label{tab:appcomparison}
\end{table}

\todo{Explain the different features we would like to take inspiration 
from.}

