\section{Existing solutions}\label{sec:exist}

During the early stages of the project we began searching for existing solutions to draw inspiration to the design and implementation phases. All members of the group have previously had experience using different recipe sites on the Internet, and we  think that the existing solutions lack functionality. In this section we review and discuss the existing solutions and draw some good practices from these.

\subsection{Supercook}
Supercook\cite{supercook} is a website with focus on searching for recipes based on their ingredients. The user is limited to enter a set of ingredients which is defined by Supercook. When adding ingredients to a search the user is aided by autocompletion and a word cloud with ingredients.

The word cloud changes according to the type of ingredients you have entered, e.g. if you enter ``vanilla'' the word cloud will change to contain typical cake ingredients. It is possible to apply restrictions to your search, e.g. ``I don't eat meat'' and all recipes with meat will be excluded. A search on Supercook will result in a prioritised list of links to recipes from several online cookbooks. A screenshot of the website is shown in \autoref{fig:supercook}.

By evaluating the search results of Supercook we have deduced what we believe are the steps used to generate the search results:
\begin{enumerate}
	\item Remove recipes without matching ingredients.
	\item Sort by most matching ingredients.
	\item Sort by least missing ingredients.
\end{enumerate}
We have also noticed some quirks in the way it counts matching ingredients. It seems like the search algorithm counts the number of occurrences of the entered ingredients in the recipes. When entering ``bell pepper'' and ``olive oil'', recipes which contain both red, green, and yellow pepper are prioritised higher than recipes which actually contains both of the entered ingredients. It also appear as if ingredients in the recipes are mapped to the recipes defined by Supercook. An example of that is that ``flour'' appear to be mapped to ``flour'', ``wheat'', ``oat'', etc. This also appear to have the shortcoming that recipes containing e.g. ``wheat flour'' are prioritised as if they have two matching ingredient when the user entered ``flour''.

When making a search containing ``eggs'', ``garlic'', ``ground beef'', and ``onion'', we get a result which showcases a drawback of the sorting done by Supercook. The first 94 recipes are simple recipes using only few of the entered ingredients, including 24 recipes on how to boil an egg. Recipe number 95 is a recipe for meatballs and is using all entered ingredients but it also needs ``bread crumbs''. The reason why the first 94 recipes are prioritised highest is because they do not need any other ingredients than the ingredients entered.
\begin{figure}[H]
\centering
\fbox{\includegraphics[width=\linewidth]{img/screenshots/supercook.png}}
\caption{The inferface of Supercook.}
\label{fig:supercook}
\end{figure}

\subsection{Allthecooks}
Allthecooks is one of the most downloaded \cite{allthecooks-googleplay} Android applications that is related to the search: ``recipe''. The application has a great design that follows the Android guidelines\cite{guidelines-appstructure} which makes it intuitive and easy to navigate. The detailed display of a recipe is especially well implemented, the user has everything on a single page, and can easily see the needed ingredients and the required steps to cook the recipe. The user can also, by a press of a button, toggle between different measuring units, or add the selected ingredient to a shopping list. Screenshots of the application is shown in \autoref{fig:allthecooks-menu}, \ref{fig:allthecooks-detail1}, \ref{fig:allthecooks-detail2}, and \ref{fig:allthecooks-detail3}.

The recipes of Allthecooks are user created, meaning they are created and uploaded by users. The photos are also provided by users, this can be dangerous if the photos are not checked for content not suitable for the application's audience. An advantage of user created recipes is that the applications content is growing with the audience of the application.
The search is free-text based, which means that opposite the Supercook web application it is hard to find recipes based on ingredients. You are able to apply filters to your search to remove recipes that contain certain ingredients.\todo{det var vist ikke helt rigtigt at supercook ikke have free-text}
\twofigs{screenshots/menu.png}{Menu of Allthecooks}{fig:allthecooks-menu}{screenshots/rainbowcake-1.png}{Detail display of a recipe}{fig:allthecooks-detail1}
\twofigs{screenshots/rainbowcake-2.png}{Buttons for different features}{fig:allthecooks-detail2}{screenshots/rainbowcake-3.png}{Directions for the recipe}{fig:allthecooks-detail3}

\subsection{BigOven}
BigOven is also one of the most downloaded \cite{bigoven-googleplay} Android application that is related to ``recipe''. The application's navigation and design does not follow the Android guidelines\cite{guidelines-appstructure} and therefore the application's navigation can be quite confusing to an Android user.

The application main page is cluttered and presents the user with many functionalities, see \autoref{fig:mainpage-bigoven}. It is possible to make both an ingredient search and a recipe search. The ingredient search is limited to only three ingredients and the algorithm finds only the recipes where they all are included. The ingredient search is based on free-text, meaning the user does not have the aid of autocompletion, like with Supercook. If the user inputs three ingredients that have no real relation like: ``beef'', ``cake-mix'', ``salmon'', the ingredient search would find no results, the ingredient search excludes everything that does not contain all three ingredients. Like Allthecooks the user can add the different ingredients to a shopping list, save the recipe to favourites, and alternate between the metric and imperial system.

A cool feature that is unique to the BigOven application is the menu-cards, see \autoref{fig:menucards-bigoven}.
\twofigs{screenshots/mainpage-bigoven.png}{The main page of BigOven}{fig:mainpage-bigoven}{screenshots/menucards-bigoven.png}{Menu Cards from BigOven}{fig:menucards-bigoven}
It is a small collection of different recipes that together builds to a meal, e.g. ``steak'', ``fries'', and ``green beans''. Each meal is bound to a day, which means that the user can create a menu-card that can contain a meal for each day of the week, or more. 

The recipes of BigOven is also, like Allthecooks, user created, which provides the same risks and benefits as explained in the previous section.
The BigOven application is free, but you can buy Pro-features that exclude advertisers from the application and unlocks more functionality in the application.

\subsection{Comparison}
These three applications have been chosen for comparison since they vary in focus, design, navigation, and features. Allthecooks focus on design and navigation, where as BigOven has focus on features. Supercook is a bit different since it not an Android application, but it was included in the comparison since it had a unique set of features, such as the word cloud, and the ingredient search.
Allthecooks and BigOven share a lot of features, they both have recipe search, shopping list, and saving of recipes. Though BigOven has some features that are unique, such as ingredient search, and menu-cards. A noticeable thing about BigOven is that the experience of using the application could be improved significantly, both in its design and performance.

BigOven and SuperCook both has ingredient search, but there is a big difference in the two searches. In the BigOven application you are only able to enter three ingredients, the search finds all the recipes that include all three ingredients. BigOven also uses free-text as an input method, whereas SuperCook aids the user with autocompletion. In SuperCook the user is also able to enter as many ingredients as necessary.
A comparison between the existing solutions can be seen in \ref{tab:appcomparison}.
\begin{table}[H]
\centering
\begin{tabular}{|>{\bfseries}l|l|l|l|}
\hline
 & \textbf{Supercook} & \textbf{Allthecooks} & \textbf{BigOven} \\
\hline
Platform & Website & Android/Website & Android/Website \\
\hline
Cost & Free & Free & Free \& Paid \\
\hline
Recipe search & Yes & Yes & Yes  \\
\hline
Ingredient search & Yes & No & Yes \\
\hline
Recipe origin & Web crawl & User defined & User defined \\
\hline
Shopping list & Yes & Yes & Yes \\
\hline
Menu planner & No & Yes & Yes \\
\hline
Saving recipes & Yes & Yes & Yes \\
\hline
Ingredient filter & Yes & Yes & Yes \\
\hline
Sharing of shopping list & No & E-mail/SMS & E-mail \\
\hline
Sharing of recipe & No & App/SMS/E-mail & App/E-mail \\
\hline
\end{tabular}
\caption{Application comparison}
\label{tab:appcomparison}
\end{table}

