During the early stages of the project we began searching for existing solution to draw inspiration to the design and implementation phases. all members of the group have previously had experience using different recipe sites on the Internet, and we all think that the different site lacks functionality. But good ideas and can be drawn from different solutions and we will in this section review and discuss these existing solutions.  

\section{Supercook}
The best searching recipe website we found is called supercook.com\cite{supercook}. We like this website because you can search for recipes by ingredients, this can be on a single ingredient or multiple. The search-bar also gives helps the user auto-complete ingredients, for faster and less error prone typing. 
The user is only able to enter ingredients in the search-bar, meaning it is not possible to do a free-text search on the recipe you already know. The website's use is to discover or find recipes with already bought ingredients, or ingredients you plan to buy. 
The website search uses a very simple method for finding recipes, normally it get the good result, but its sorting of results is weird. The sorting algorithm sorts the results in what you have all the ingredients for highest, which can result in some unwanted recipes. \todo{explain the cloud, and add pictures of the search-bar}

\section{Allthecooks}
Allthecooks is one of the most downloaded \cite{allthecooks-googleplay} Android application that is related to food and cooking. The application has a great design, it is intuitive and easy to navigate. The detailed display of a recipe is especially well implemented, the user has everything on a single activity, and can easily see the needed ingredients and the required steps to cook the recipe. The user can also by a press of a button calculate between different measuring units, or add the selected ingredient to a shopping list. Screenshot of the application is shown on Figures \ref{fig:allthecooks-menu}, \ref{fig:allthecooks-detail1}, \ref{fig:allthecooks-detail2}, \ref{fig:allthecooks-detail3}. The application is very elegant in its design, the search is free-text based, which means that opposite the Supercook web application it is hard to find recipes with multiple ingredients.
\twofigs{screenshots/menu.png}{Menu of Allthecooks}{fig:allthecooks-menu}{screenshots/rainbowcake-1.png}{Detail display of a recipe}{fig:allthecooks-detail1}
\twofigs{screenshots/rainbowcake-2.png}{Buttons for different features}{fig:allthecooks-detail2}{screenshots/rainbowcake-3.png}{Directions for the recipe}{fig:allthecooks-detail3}

\section{Review of existing solution}
\todo{find a better headline.} After the review of different applications, of both web and android applications. We believe that we can create an application with improved features, such as a better search algorithms, easy navigation, and an elegant design. We like the idea of being able to search for recipes by ingredients, but we wont exclude the possibility to also directly search for recipes. We will also provide the user with different input styles; type(free text and ingredient restricted) input, and tile input. Tile input is basically the same as ingredient search, but instead of typing the name of the ingredient the user selects the ingredient by use of tiles. The ingredient are divided into categories and maybe even subcategories, to make the navigation more sensible, otherwise the user would be presented with a too many possibilities and the screen would also be cluttered.  
